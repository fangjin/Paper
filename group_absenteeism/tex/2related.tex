%\section{Related Work}
\paragraph{Burst event detection:}
Lappas et al.~\cite{lappas2009burstiness} define the temporal burstiness of a term using
the term frequency within one time interval, divided by the whole test period, and then 
minus the average term frequency. The authors further generalized their approach to detecting spatiotemporal bursts of certain terms in geocoded document streams~\cite{lappas2012spatiotemporal}. They use axis-oriented rectangles to constrain the selected regions to be spatially connected. Garcia-Gasulla et al.~\cite{garcia2014detection} discovered events based on collaborative social network data. Sakaki at al.~\cite{sakaki2010earthquake} considered Twitter users as social sensors to detect earthquakes. Rozenshtein et al.~\cite{rozenshtein2014event} detect events by capturing the compactness of a graph.

\paragraph{Spatial clustering:}
Spatial clustering groups the objects in a spatial data set and identifies contiguous regions in the space of the spatial attributes. Ng and Han~\cite{ng1994cient} proposed a clustering method called CLARANS based on randomized search that does not explicitly handle noise and requires users to predefine the number 
of clusters. Ester et al.~\cite{ester1996density} developed the density-based DBSCAN method
that is able to detect free shape clusters and does not require users to define the number of clusters. Cao et al.~\cite{cao2013analyzing} propose a supervised approach by using interestingness functions which assess the quality of spatial clusters based on uniformity measures to capture a domain expert's notion of uniformity. Recently, Rodriguez and Laio~\cite{rodriguez2014clustering} proposed a fast search clustering method to identify density peaks, which can remove outliers at the same time and does not require any predefined parameters.

\paragraph{Graph partitioning:}
The problem of graph partitioning consists of dividing the vertices in $g$ groups of predefined size, such that the number of edges lying between the groups is minimal~\cite{fortunato2010community}. The most popular algorithm is built around the idea of using centrality indices to find community boundaries~\cite{girvan2002community}. 
\iffalse
This algorithm is an iterative process with three components in each iteration: 1. Computation of the centrality for all edges;
2. Removal of edge with largest centrality: in case of ties with other edges, one of them is picked at random;
3. Recalculation of centralities for all the edges.
\fi
Karypis et al. developed a set of algorithms for partitioning graphs based on the the proposed multilevel recursive-bisection, multilevel k-way, and multi-constraint partitioning schemes~\cite{karypis2003multi, abou2006multilevel, lasalle2013multi}.

\noindent
As stated earlier our goal, distinct from burstiness detection, is to
design efficient algorithms for maximizing absenteeism score functions over possible groups.
Two different scenarios are considered. When data objects have geographic coordinates, we consider spatial clustering techniques to enforce spatial coherence of data objects in a possible absenteeism group. When data objects are organized as vertices in a general graph, we apply graph partitioning techniques to enforce graph coherence of data objects in a possible absenteeism group.
